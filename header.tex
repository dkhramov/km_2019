% Предотвращает предупреждения “Font shape `OT1/cmss/m/n' in size <4> not available” и “Size substitutions with differences”
\let\Tiny=\tiny
\let\TINY=\tiny
% Предотвращает появление предупреждений hyperref 
\hypersetup{unicode=true}
%
\setbeamercovered{transparent}
% отображение номера в нижнем правом углу слайда
\setbeamerfont{footline}{size=\normalsize}
\makeatletter % в нижней части слайда отображается только его номер 
\setbeamertemplate{footline}
{
  \leavevmode%
  \hbox{%
  \begin{beamercolorbox}[wd=\paperwidth,ht=2.25ex,dp=1ex,right]{date in head/foot}%
    \usebeamerfont{date in head/foot}\insertframenumber\hspace*{2ex} 
  \end{beamercolorbox}}%
  \vskip0pt%
}
\makeatother
% Удаляем слово 'Figure' из заголовка рисунка
%\setbeamertemplate{caption}{\insertcaption\par}
%\captionsetup{labelformat=empty,labelsep=none}
%
\DeclareMathOperator*{\argmin}{argmin}
% Две колонки
\newcommand{\bcols}{\begin{columns}}
\newcommand{\ecols}{\end{columns}}
